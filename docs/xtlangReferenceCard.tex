\documentclass[10pt, twoside]{article}   	% use "amsart" instead of "article" for AMSLaTeX format
\usepackage[top=0.5in,bottom=0.5in,left=0.5in,right=0.5in]{geometry}                		% See geometry.pdf to learn the layout options. There are lots.
\geometry{letterpaper}                   		% ... or a4paper or a5paper or ... 
\geometry{landscape}                		% Activate for for rotated page geometry

\usepackage[parfill]{parskip}    		% Activate to begin paragraphs with an empty line rather than an indent
%\usepackage{amssymb}
\usepackage{setspace}

\usepackage{color}
%Extempore code highlighting colors
\definecolor{xtLangTypecolor}{RGB}{85,142,40}	%Green
\definecolor{xtLangColor}{RGB}{46,111,253}	%Blue
\definecolor{schemeVerbColor}{RGB}{168,24,75} % Red
\definecolor{xtLangSchemeVerbColor}{RGB}{159,0,197} % Purple

\usepackage{multicol}

\usepackage{titlesec}
\titlespacing*{\section}{0pt}{0pt}{0pt}
\titlespacing*{\subsection}{0pt}{0pt}{0pt}
%\titlespacing*{\paragraph}{0pt}{0.5\baselineskip}{1em}
\titlespacing*{\paragraph}{0pt}{0pt}{0pt}

\usepackage{textcomp}

%\leftskip=0pt plus .5fil
\rightskip=0pt plus -.5fil
\parfillskip=0pt plus .5fil

\title{Extempore Reference Card}
\author{Neil Tiffin}
\date{\today}				% Activate to display a given date or no date

% Turn off header and footer
\pagestyle{empty}
\righthyphenmin=64
\lefthyphenmin=64 
 
% \null or \hphantom{.} to have object to fill against.
% \hfill or \hspace{ \stretch{1} } to fill space in middle of line
% Math Mode, \; thick space, \: medium space, \, thin space, \! negative thin space



\begin{document}
\newcommand{\xtType}[1]{\texttt\scriptsize{\textcolor{xtLangTypecolor}{ #1}}\normalsize}
\newcommand{\xtFunc}[1]{\textcolor{xtLangColor}{#1}}
\newcommand{\schemeVerb}[1]{\textcolor{schemeVerbColor}{#1}}
\newcommand{\xtSchemeVerb}[1]{\textcolor{xtLangSchemeVerbColor}{#1}}
\begin{multicols}{3}

\begin{center}
     \Large{\textbf{Extempore Reference Card}}
\end{center}
\section*{xtlang Types}
\begin{spacing}{1.5}
Boolean							\hfill \xtType{i1} \\
Boolean pointer						\hfill \xtType{i1*} \\
Character							\hfill \xtType{i8} \\
Strings {\scriptsize(are null terminated)}	\hfill \xtType{i8*} \\
Integer							\hfill \xtType{i32} \\
Integer pointer						\hfill \xtType{i32*} \\
Long integer (default)				\hfill \xtType{i64} \\
Long integer pointer					\hfill \xtType{i64*} \\
32 bit float							\hfill \xtType{float} \\
32 bit float	 pointer					\hfill \xtType{float*} \\
64 bit double (default)				\hfill \xtType{double} \\
64 bit double pointer					\hfill \xtType{double*} \\
Variable a1 declared as pointer to double \hfill a1:\xtType{double*} \\
C type void*						\hfill \xtType{i8*} \\
Tuple of 2 i8's						\hfill \xtType{\textless  i8, i8\textgreater} \\
Pointer to tuple of double and i32		\null \hfill \xtType{\textless double, i32\textgreater *} \\
Array of length and type 				\hfill \xtType{$\vert$length, type$\vert$} \\
Pointer to an array of 4 double			\hfill \xtType{$\vert$4, double$\vert$*} \\
Vector of length and type				\hfill \xtType{/length, type/} \\
Pointer to a vector of 4 floats			\hfill \xtType{/4,float/*} \\
Closure							\hfill \xtType{[return type, arg type,$\:\mathit{. . .}$]}\\
Pointer to closure which takes 2 i32 arguments and returns i64\hfill \xtType{[i64, i32, i32]*} \\
\scriptsize
{\raggedright
`*' is not a deference operator, it is part of the type label.\par}
\normalsize
\vspace{-5mm}
\end{spacing}


\section*{Custom xtlang Types}
Define type for LLVM compiler. \\
\null\hfill (\schemeVerb{bind-type} name type) \\
Tell pre-processor to textually substitute type for name before compiling. \hfill (\schemeVerb{bind-alias} name type) \\
Define value.\hfill (\schemeVerb{bind-val} name type value)

\section*{xtlang Coercion Functions}
\begin{multicols}{3}
(\xtFunc{i1toi64}~\xtType{i1})
(\xtFunc{i64toi1}~\xtType{i64})
(\xtFunc{ftoi64}~\xtType{float})
(\xtFunc{i64tod}~\xtType{i64})
(\xtFunc{dtoi64}~\xtType{double})
(\xtFunc{i32toptr}~\xtType{i32})
(\xtFunc{ptrtoi64}~\xtType{ptr})
(\xtFunc{dtof}~\xtType{double}) \\
etc.
\end{multicols}


\section*{Common xtlang \& Scheme}
The following functions are available in both Scheme and xtlang and should follow R5RS, except as noted.\\
if $c_{1}$ then $e_{2}$ else $e_{3}$
					\hfill(\xtSchemeVerb{if} $c_{1}\:e_{2}\:e_{3}$) \\
Assign $s_1 = e_1$, eval expr
					\hfill(\xtSchemeVerb{let} ( ($s_1\:e_1$)$\:\mathit{. . .}$) expr) \\
Sequence				\hfill(\xtSchemeVerb{begin} \\
Boolean true			\hfill\xtSchemeVerb{\#t} \\
Boolean false			\hfill\xtSchemeVerb{\#f} \\
Boolean and			\hfill(\xtSchemeVerb{and} $n_1\:n_2\:\mathit{. . .}$) \\
Boolean or			\hfill(\xtSchemeVerb{or} $n_1\:n_2\:\mathit{. . .}$) \\
Boolean not			\hfill(\xtSchemeVerb{not} $o_1$) \\
Anonymous function		\hfill(\xtSchemeVerb{lambda} ($a_{1}\:\mathit{...}$) expr)\\
Multiply				\hfill(\xtSchemeVerb{*} $n_1\:\mathit{. . .}$) \\
Divide				\hfill(\xtSchemeVerb{/} $n_1\:n_2\:\mathit{. . .}$) \\
Add					\hfill(\xtSchemeVerb{+} $n_1\:\mathit{. . .}$) \\
Subtract				\hfill(\xtSchemeVerb{-} $n_1\:n_2\:\mathit{. . .}$) \\
Equality				\hfill(\xtSchemeVerb{=} $n_1\:n_2$) \\
Less Than				\hfill(\xtSchemeVerb{\textless} $n_1\:n_2$) \\
Greater Than			\hfill(\xtSchemeVerb{\textgreater} $n_1\:n_2$) \\
Not equal				\hfill(\xtSchemeVerb{\textless\textgreater} $n_1\:n_2$) \\
Remainder			\hfill(\xtSchemeVerb{modulo} $n_1\:n_2$) \\
Is null				\hfill(\xtSchemeVerb{null?} $n_1$) \\
Assignment			\hfill(\xtSchemeVerb{set!} $s_1\:e_1$) \\
Loop {\scriptsize(from common lisp)}\hfill(\xtSchemeVerb{dotimes} ($s_1\:e_1$) expr)\\
Conditional			\hfill(\xtSchemeVerb{cond} ($c_{1}\:e_{2}$) $\mathit{. . .}$ (\xtSchemeVerb{else} $e_{n}$)) \\
\scriptsize
{\raggedright
$a_x$ argument,
$c_x$ boolean expression,
$e_x$ general expression, \\
$n_x$ numeric expression, 
$o_x$ object,
$s_x$ symbol\par}
\normalsize


\section*{xtlang Memory Allocation}
Allocate on stack.\hfill(\xtFunc{salloc} size) \\
Allocate on heap.\hfill(\xtFunc{halloc} size) \\
Create memory zone.\hfill(\xtFunc{memzone} size ...) \\
Allocate in zone.\hfill(\xtFunc{zalloc} size) \\
Push zone.\hfill(\xtFunc{push\_zone}) \\
Pop zone.\hfill(\xtFunc{pop\_zone}) \\
long int *a1 = 2; \\
\hphantom{.}\hfill (\xtSchemeVerb{let} ((a1:\xtType{i64*} (\xtFunc{salloc}))) (\xtFunc{pset!} a1 0 2) ...) \\
int a1[4]; 	\hfill(a1:\xtType{i32*} (\xtFunc{zalloc} 4))


\section*{xtlang Pointers \& Aggregrates}
Deference pointer	\hfill(\xtFunc{pref}~ptr$_1$~offset$_1$) \\
Deference tuple	\hfill(\xtFunc{tref}~ptr$_1$~offset$_1$) \\
Deference array	\hfill(\xtFunc{aref}~ptr$_1$~offset$_1$) \\
Deference vector	\hfill(\xtFunc{vref}~ptr$_1$~offset$_1$) \\
Set pointer		\hfill(\xtFunc{pset!}~ptr$_1$~offset$_1$~$p_1$) \\
Set tuple			\hfill(\xtFunc{tset!}~ptr$_1$~offset$_1$~$t_1$) \\
Set array			\hfill(\xtFunc{aset!}~ptr$_1$~offset$_1$~$a_1$) \\
Set vector			\hfill(\xtFunc{vset!}~ptr$_1$~offset$_1$~$v_1$) \\
Fill pointer			\hfill(\xtFunc{pfill!}~ptr$_1$~$e_1$,~...~$e_n$) \\
Fill tuple			\hfill(\xtFunc{tfill!}~ptr$_1$~$e_1$,~...~$e_n$) \\
Fill array			\hfill(\xtFunc{afill!}~ptr$_1$~$e_1$,~...~$e_n$) \\
Fill vector	 		\hfill(\xtFunc{vfill!}~ptr$_1$~$e_1$,~...~$e_n$) \\
Get address pointer	\hfill(\xtFunc{pref-ptr}~ptr$_1$~offset$_1$) \\
Get address tuple	\hfill(\xtFunc{tref-ptr}~ptr$_1$~offset$_1$) \\
Get address array	\hfill(\xtFunc{aref-ptr}~ptr$_1$~offset$_1$) \\
Get address vector	\hfill(\xtFunc{vref-ptr}~ptr$_1$~offset$_1$) \\
\scriptsize
{\raggedright
$a_x$ array.
ptr$_x$ i1*, i8*, i16*, i32*, float*, double* pointers.
offset$_x$ natural number.
$e_x$ expression.
$n_x$ float, double, i1, i8, i32, i64 expression.
$t_x$ tuple.
$v_x$ vector.
\par}
\normalsize


\section*{xtlang Core Functions}
Uses boost random() or C rand() [0.0..1.0]  \\
\hphantom{.}\hfill\xtType{double}~(\xtFunc{random}) \\
\hphantom{.}\hfill(\xtFunc{begin} $e_1$ $e_2$ ...) \\
\hphantom{.}\hfill\xtType{null} \\
\hphantom{.}\hfill(\xtFunc{now}) \\
void \\
\hphantom{.}\hfill(\xtFunc{lambdas} ?) \\
\hphantom{.}\hfill(\xtFunc{lambdaz} ?) \\
\hphantom{.}\hfill(\xtFunc{lambdah} ?) \\
\hphantom{.}\hfill(\xtFunc{printf}~format:\xtType{i8*}~$e_1$~$e_2$...) \\
\hphantom{.}\hfill(\xtFunc{sprintf}~str:\xtType{i8*}~format:\xtType{i8*}~$e_1$~$e_2$...)\\
Callback\hfill(\xtFunc{callback} time name args ...) \\
Schedule {\scriptsize(same as callback)}\hfill(\xtFunc{schedule} time name args ...)  \\
cast $\vert$ bitcast\hfill(\xtFunc{bitcast} ?)\\
convert $\vert$ bitconvert\hfill(\xtFunc{bitconvert} ?) \\
\& $\vert$ bitwise-and\hfill(\xtFunc{bitwise-and}~$n_1$~$n_2$~...) \\
bor $\vert$ bitwise-or\hfill(\xtFunc{bitwise-or}~$n_1$~$n_2$~...) \\
$\wedge$ $\vert$ bitwise-eor\hfill(\xtFunc{bitwise-eor}~$n_1$~$n_2$~...)\\
\textless\textless $\vert$ bitwise-shift-left\hfill(\xtFunc{bitwise-shift-left}~$n_1$~$n_2$) \\
\textgreater\textgreater $\vert$ bitwise-shift-right\hfill(\xtFunc{bitwise-shift-right}~$n_1$~$n_2$)\\
\textasciitilde~$\vert$~bitwise-not\hfill(\xtFunc{bitwise-not}~$n_1$) \\
\scriptsize
{\raggedright
$e_x$ expression,
$n_x$ float, double, i1, i8, i32, i64 expression\par}
\normalsize

\section*{Extempore Scheme}
{\raggedright
Originally from TinyScheme v1.35. Implements most of R5RS except macro, which is a common lisp style macro.\par}


\section*{Time}
Now.	\hfill(\xtFunc{now}) \\
Second constant.	\hfill*second* \\
Minute constant.	\hfill*minute* \\
Hour constant.	\hfill*hour*


\section*{General Functions}
TODO

\section*{Music Functions}
 \hfill
(\xtFunc{define-instrument} name note\_c effect\_c ...) \\
Note closure \\
\null\hfill[[output,time,channel,freq,volume]*]* \\
Effect closure \\  
\null\hfill[output,input,time,channel,data*]* \\
\hphantom{.}\hfill(\xtFunc{play-note} time inst pitch volume duration)\par
TODO

\section*{Debugging}
TODO

\section*{xtlang C bindings math.h c99}
\begin{multicols}{3}
(\xtFunc{cos}~$r_1$) \\
(\xtFunc{tan}~$r_1$) \\
(\xtFunc{sin}~$r_1$)
(\xtFunc{cosh}~$r_1$)
(\xtFunc{tanh}~$r_1$)
(\xtFunc{sinh}~$r_1$)
(\xtFunc{acos}~$r_1$)
(\xtFunc{asin}~$r_1$)
(\xtFunc{atan}~$r_1$)
(\xtFunc{atan2}~$r_1$~$r_2$)
(\xtFunc{ceil}~$r_1$)
(\xtFunc{floor}~$r_1$)
(\xtFunc{exp}~$r_1$)
(\xtFunc{fmod}~$r_1$~$r_2$)
(\xtFunc{pow}~$r_1$~$r_2$)
(\xtFunc{log}~$r_1$)
(\xtFunc{log2}~$r_1$)
(\xtFunc{log10}~$r_1$)
(\xtFunc{sqrt}~$r_1$)
(\xtFunc{fabs}~$r_1$)
%c99
(\xtFunc{acosh}~$r_1$)
(\xtFunc{asinh}~$r_1$)
(\xtFunc{atanh}~$r_1$)
(\xtFunc{cbrt}~$r_1$)
(\xtFunc{copysign}~$r_1$~$r_2$)
(\xtFunc{erf}~$r_1$)
(\xtFunc{erfc}~$r_1$)
(\xtFunc{exp2}~$r_1$)
(\xtFunc{expm1}~$r_1$)
(\xtFunc{fdim}~$r_1$~$r_1$)
(\xtFunc{fma}~$r_1$~$r_2$~$r_3$)
(\xtFunc{fmax}~$r_1$~$r_1$)
(\xtFunc{fmin}~$r_1$~$r_1$)
(\xtFunc{hypot}~$r_1$~$r_1$)
(\xtFunc{ilogb}~$r_1$)
(\xtFunc{lgamma}~$r_1$)
\xtType{i64}~(\xtFunc{llrint}~$r_1$)
\xtType{i64}~(\xtFunc{lrint}~$r_1$)
\xtType{i32}~(\xtFunc{rint}~$r_1$)
\xtType{i64}~(\xtFunc{llround}~$r_1$)
\xtType{i32}~(\xtFunc{lround}~$r_1$)
(\xtFunc{log1p}~$r_1$)
\xtType{i32}~(\xtFunc{logb}~$r_1$)
(\xtFunc{nan}~\xtType{i8*})
(\xtFunc{nearbyint}~$r_1$)
(\xtFunc{nextafter}~$r_1$~$r_2$)
(\xtFunc{nexttoward}~$r_1$~$r_2$)
(\xtFunc{remainder}~$r_1$~$r_2$)
(\xtFunc{remquo}~$r_1$~$r_2$~\xtType{i8*})
(\xtFunc{round}~$r_1$)
(\xtFunc{scalbn}~$r_1$,~\xtType{i32})
(\xtFunc{tgamma}~$r_1$)
(\xtFunc{trunc}~$r_1$) \\
\end{multicols}
\scriptsize
{\raggedright
$r_x$ double or float. Return type same as args, except as noted.\par}
\normalsize


\section*{xtlang C bindings stdio.h}
\begin{multicols}{2}
\xtType{void}~(\xtFunc{clearerr}~\xtType{i8*})
\xtType{i8*}~(\xtFunc{ctermid}~\xtType{i8*})
\xtType{i32}~(\xtFunc{fclose}~\xtType{i8*})
\xtType{i8*}~(\xtFunc{fdopen}~\xtType{i32,i8*})
\xtType{i32}~(\xtFunc{feof}~\xtType{i8*})
\xtType{i32}~(\xtFunc{ferror}~\xtType{i8*})
\xtType{i32}~(\xtFunc{fflush}~\xtType{i8*})
\xtType{i32}~(\xtFunc{fgetc}~\xtType{i8*})
\xtType{i8*}~(\xtFunc{fgets}~\xtType{i8*,i32,i8*})
\xtType{i32}~(\xtFunc{fileno}~\xtType{i8*})
\xtType{void}~(\xtFunc{flockfile}~\xtType{i8*})
\xtType{i8*}~(\xtFunc{fopen}~\xtType{i8*,i8*})
\xtType{i32}~(\xtFunc{fputc}~\xtType{i32,i8*})
\xtType{i32}~(\xtFunc{fputs}~\xtType{i8*,i8*})
\xtType{i64}~(\xtFunc{fread}~\xtType{i8*,i64,i64,i8*})
\xtType{i8*}~(\xtFunc{freopen}~\xtType{i8*,i8*,i8*})
\xtType{i32}~(\xtFunc{fseek}~\xtType{i8*,i64,i32})
\xtType{i64}~(\xtFunc{ftell}~\xtType{i8*})
\xtType{i32}~(\xtFunc{ftrylockfile}~\xtType{i8*})
\xtType{void}~(\xtFunc{funlockfile}~\xtType{i8*})
\xtType{i64}~(\xtFunc{fwrite}~\xtType{i8*,i64,i64,i8*})
\xtType{i32}~(\xtFunc{getc}~\xtType{i8*})
\xtType{i32}~(\xtFunc{getchar})
\xtType{i32}~(\xtFunc{getc\_unlocked}~\xtType{i8*})
\xtType{i32}~(\xtFunc{getchar\_unlocked})
\xtType{i8*}~(\xtFunc{gets}~\xtType{i8*})
\xtType{i32}~(\xtFunc{getw}~\xtType{i8*})
\xtType{i32}~(\xtFunc{pclose}~\xtType{i8*})
\xtType{void}~(\xtFunc{perror}~\xtType{i8*})
\xtType{i8*}~(\xtFunc{popen}~\xtType{i8*,i8*})
\xtType{i32}~(\xtFunc{putc}~\xtType{i32,i8*})
\xtType{i32}~(\xtFunc{putchar}~\xtType{i32})
\xtType{i32}~(\xtFunc{putc\_unlocked}~\xtType{i32,i8*})
\xtType{i32}~(\xtFunc{putchar\_unlocked}~\xtType{i32})
\xtType{i32}~(\xtFunc{puts}~\xtType{i8*})
\xtType{i32}~(\xtFunc{putw}~\xtType{i32,i8*})
\xtType{i32}~(\xtFunc{remove}~\xtType{i8*})
\xtType{i32}~(\xtFunc{rename}~\xtType{i8*,i8*})
\xtType{void}~(\xtFunc{rewind}~\xtType{i8*})
\xtType{void}~(\xtFunc{setbuf}~\xtType{i8*,i8*})
\xtType{i32}~(\xtFunc{setvbuf}~\xtType{i8*,i8*,i32,i64})
\xtType{i8*}~(\xtFunc{tempnam}~\xtType{i8*,i8*})
\xtType{i8*}~(\xtFunc{tmpfile})
\xtType{i8*}~(\xtFunc{tmpnam}~\xtType{i8*})
\xtType{i32}~(\xtFunc{ungetc}~\xtType{i32,i8*})
\xtType{i32}~(\xtFunc{llvm\_printf}$^1$~\xtType{i8*},~...)
\xtType{i32}~(\xtFunc{llvm\_sprintf}$^1$~\xtType{i8*,i8*},~...)
\end{multicols}
\scriptsize
{\raggedright
$^1$~Can't be standard printf because of variable args\par}
\normalsize


\section*{xtlang C bindings stdlib.h}
\begin{multicols}{2}
\xtType{i8*}~(\xtFunc{malloc}~\xtType{i64})
\xtType{void}~(\xtFunc{free}~\xtType{i8*})
\xtType{i8*}~(\xtFunc{malloc16}~\xtType{i64})
\xtType{void}~(\xtFunc{free16}~\xtType{i8*})
\xtType{i8*}~(\xtFunc{getenv}~\xtType{i8*})
\xtType{i32}~(\xtFunc{system}~\xtType{i8*}) \\
\end{multicols}


\section*{xtlang C bindings string.h}
\begin{multicols}{2}
\xtType{double}~(\xtFunc{atof}~\xtType{i8*})
\xtType{i32}~(\xtFunc{atoi}~\xtType{i8*}) \\
\xtType{i64}~(\xtFunc{atol}~\xtType{i8*})
\xtType{i8*}~(\xtFunc{memccpy}~\xtType{i8*,i8*,i32,i64})
\xtType{i8*}~(\xtFunc{memchr}~\xtType{i8*,i32,i64})
\xtType{i32}~(\xtFunc{memcmp}~\xtType{i8*,i8*,i64})
\xtType{i8*}~(\xtFunc{memcpy}~\xtType{i8*,i8*,i64})
\xtType{i8*}~(\xtFunc{memmove}~\xtType{i8*,i8*,i64})
\xtType{i8*}~(\xtFunc{memset}~\xtType{i8*,i32,i64})
\xtType{i8*}~(\xtFunc{strcat}~\xtType{i8*,i8*})
\xtType{i8*}~(\xtFunc{strchr}~\xtType{i8*,i32})
\xtType{i32}~(\xtFunc{strcmp}~\xtType{i8*,i8*})
\xtType{i32}~(\xtFunc{strcoll}~\xtType{i8*,i8*})
\xtType{i8*}~(\xtFunc{strcpy}~\xtType{i8*,i8*})
\xtType{i64}~(\xtFunc{strcspn}~\xtType{i8*,i8*})
\xtType{i8*}~(\xtFunc{strdup}~\xtType{i8*})
\xtType{i8*}~(\xtFunc{strerror}~\xtType{i32})
\xtType{i64}~(\xtFunc{strlen}~\xtType{i8*})
\xtType{i8*}~(\xtFunc{strncat}~\xtType{i8*,i8*,i64})
\xtType{i32}~(\xtFunc{strncmp}~\xtType{i8*,i8*,i64})
\xtType{i8*}~(\xtFunc{strncpy}~\xtType{i8*,i8*,i64})
\xtType{i8*}~(\xtFunc{strpbrk}~\xtType{i8*,i8*})
\xtType{i8*}~(\xtFunc{strrchr}~\xtType{i8*,i32})
\xtType{i64}~(\xtFunc{strspn}~\xtType{i8*,i8*})
\xtType{i8*}~(\xtFunc{strstr}~\xtType{i8*,i8*})
\xtType{i8*}~(\xtFunc{strtok}~\xtType{i8*,i8*})
\xtType{i8*}~(\xtFunc{strtok\_r}~\xtType{i8*,i8*,i8**})
\xtType{i64}~(\xtFunc{strxfrm}~\xtType{i8*,i8*,i64}) \\
\end{multicols}
\scriptsize


\section*{Function Name Conventions}
`!' at end of name indicates function is destructive (e.g. mutates the arguments passed to it). \\
`\_c' at end of name indicates an xtlang closure. \\
`-' minus sign separator denotes Scheme name. \\
`\_' underscore separator denotes xtlang name. \\
`*' on both ends of variable denotes global scope. \\
`?' at the end of name denotes function returns boolean. 

\scriptsize
\flushleft
\textcolor{xtLangTypecolor}{Color denotes xtlang type.} \\
\textcolor{xtLangColor}{Color denotes xtlang only verb.} \\
\textcolor{schemeVerbColor}{Color denotes scheme only verb.} \\
\textcolor{xtLangSchemeVerbColor}{Color denotes common xtlang and Scheme verb.} \\
\vfill
\today~v1.0 \copyright 2013 Neil Tiffin \\
Permission is granted to make and distribute copies of this card provided the copyright notice and this permission notice are preserved on all copies.  Send comments and corrections to neilt@neiltiffin.com

\end{multicols}
\end{document}  